\documentclass[a4paper,11pt]{article}
\usepackage{fullpage}
\usepackage{hyperref}
\usepackage{amsmath}
\usepackage{amssymb}
\usepackage{amsthm}
\usepackage{fancybox}
\usepackage{graphicx}
\usepackage{caption}
\usepackage{subcaption}
\usepackage{float}
\usepackage{natbib}
\bibliographystyle{abbrv}

\newtheorem{corollary}{Corollary}
\newtheorem{theorem}{Theorem}
\newtheorem{lemma}{Lemma}
\newtheorem{definition}{Definition}
\newtheorem{proposition}[theorem]{Proposition}
\newtheorem{warning}[theorem]{Warning}
\newcommand{\CAT}{\textrm{CAT}}
\newcommand{\BHV}{\textrm{BHV}}
\newcommand{\aC}{\mathcal{C}}
\newcommand{\aD}{\mathcal{D}}
\newcommand{\aE}{\mathcal{E}}
\newcommand{\aK}{\mathcal{K}}
\newcommand{\aP}{\mathcal{P}}
\newcommand{\Out}{\textrm{Out}}

\begin{document}

\title{Synthetic experiments to-do list}

\author{Sakellarios Zairis$^1$, Hossein Khiabanian$^1$, Andrew J. Blumberg$^2$, and Raul Rabadan$^1$\\
\\
$^1$ Department of Systems Biology, Columbia University\\
$^2$ Department of Mathematics, UT Austin\\}
\maketitle

\section{Experiments}

This is a list of potential synthetic experiments to do and techniques
to incorporate with real data, for that matter:

\begin{enumerate}

\item Distinguishing distributions by mapping into Euclidean space, as
  in section 3.2.  Representative basic idea: take a bunch of
  landmarks, embed samples from tree space into Euclidean space by
  using the distances to these landmarks.  Then take moments there or
  somesuch.  As discussed in 3.3, we might consider using the
  Kolmogorov-Smirnov test on such an embedding.  These tests will 
  make sense in the analysis for sections 6.1 and 6.2.

\item Clustering: we probably want to test spectral clustering on our
  synthetic data as well.  And maybe dbscan.  This is all pretty easy,
  of course, given the clustering framework.  (We should also consider
  k-medoids, and can contemplate test statistics that involve the
  distance to the medoid associated to a cluster --- this is probably
  a further useful addendum to what happens in 6.1 and 6.2.)

\item Tree dimensionality reduction: we need to do a bunch of
  synthetic experiments to provide quantitative examples of the
  behavior guaranteed by the stability theorems.

\end{enumerate}

\end{document}
